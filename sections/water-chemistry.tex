\section{Water Chemistry}
 In general clean good tasting water is best. Low chlorine and low/no chloramines. High gravity musts do require 
 a certain amount of Zinc and Magnesium for yeast health, but in general those are provided by products like
 Fermaid-K and GO-FERM.
 \subsection{More Research Needed}
  Unfortunately there is not a lot a information on the impact of various salts on the mead. Especially in regard
  To fermentation dynamics and final taste characteristics. Some speculate that applying similar knowledge from 
  beer making may have a similar impact. For example perhaps a small amounts of sodium may increase sweetness 
  or a higher ratio of chlorides to sulfates may increase body. 

  In conclusion it seems that water chemistry is an understudied area of mead making.