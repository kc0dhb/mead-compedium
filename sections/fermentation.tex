\section{Fermentation}
 This is perhaps the most important aspect of mead making. You can have the best ingredients, but if you don't 
 have a healthy fermentation, your honey, and other ingedients, won't shine. This section is ordered in 
 approximate order of the magnitude of impact. The farther you read and apply, the better your meads will go. 
 The earlier topics have more ``bang for your buck'' and are  usually quite easy to implement. 

%TODO reword to sound less jolting
 Nutrient addition and yeast pitching rates are quite easy to implement. Temperature control can be done quite 
 low-tech (with manual effort) or high-tech with automation. Oxygen stones aren't incredibly cheap but they do 
 have a fair bit of benefit. The usage of pH meters, while relatively expensive, can give you quite a bit of 
 useful information on the current status of your mead.
 \subsection{Yeast}
%TODO add temperature ranges for all yeasts.
  There is a vast variety of high quality yeasts available to the average meadmaker. Highlighed below are a few 
  of the favorite more easily available ones. In general, the best results from a given yeast are usually obtained 
  when fermenting in the lower third of the yeast's fermentation temperature range.

%TODO add alcohol tolerances.
%TODO surlee spelling?
  \textbf{D-47:} A classic yeast for white wines which is often used to great effect in traditionals. 
   Be careful of letting it get to warm (above 70\textdegree F) as it throws off fusels and harsh alcohols. 
   It can Sur Lie for quite some time, adding floral characteristics.

%TODO double check spellings
%TODO and details on good for stuff
  \textbf{DV10:} A champagne isolate of a bayanus strain that produces minimal esters and phenols. Also tends to 
   not blow off more delicate honey aromas. Ferments to 18\% regularly. Good for dry meads and high alcohol sweet 
   meads. It is a low nutrient yeast, but tends to produce sulfur unless front-loaded with nutrients 
   (50/25/25 schedule or 75/25 seems to work well).

%TODO specifics?
  \textbf{71B-1122:} Narbonne yeast that can metabolize some malic acid. 
   This is good in that it can reduce the overall acidity, resulting in a more palatable mead with less aging if
   used with malic acid fruits, useful for young meads (low alcohol) or those with a fruit containing a large 
   percentage of malic acid. 
%TODO get a reference? AToE is the only one I know of that had identified this phenomenon
   Some report that using low malic acid fruits with this yeast produce a higher apparent acidity.

  \textbf{K1V-1116:} A generic fruit wine yeast. Good results have been reported in high temperature fermentation 
   environments such as in excess of 75\textdegree F.

  \textbf{Uvaferm 43:} A bayanus yeast strain that has an alcohol tolerance of 18\%+. It is commonly used for 
   high alcohol wines and tends not to blow off delicate aromas. Also commonly used with dessert wines and stuck 
   fermentations. 


 \subsection{Yeast Nutrients}
%TODO update nutrient amounts and provide a bibliography
%TODO Have both g/L and g/Gallon measurements
  Nutrients are used to make up for the lack of proper nutrients in the must, to aid in fermentation speed, or 
  to overcome a stuck fermentation. They are typically added in relatively small amounts throughout the fermentation
  process. Because of the small amount needed, adding too much can be done easily, so care should be taken when 
  calculating and adding nutrients to the must.

%TODO find out how much YAN GO-FERM provides
  \textbf{GO-FERM:} Used in rehydrating yeast. 1.25g of GO-FERM per gram of yeast with 17g grams of water. 
   Mix GO-FERM and water together and add yeast at re-hydration temperature (typically 104-109F). 
%TODO Finish sentence ``Provides 

  \textbf{DAP:} Also known as Di-ammonium phosphate, chemical name \ce{(NH_4)_2 HPO_4} is a source of inorganic 
   nitrogen, which provides YAN (Yeast Available Nitrogen) for nitrogen deficient musts (such as honey). 
   1g/L provides 210ppm of YAN. 1g/Gallon provides 50ppm of YAN.  Do not add past 2/3 sugar break. 
   The yeast cannot consume the nutrient at this stage and it will likely result in Urea type aromas and 
   flavors in the finished mead. Do not add to hydrating yeast as DAP is somewhat toxic to re-hydrating yeast.
% TODO Add notes on off flavors from too much DAP
% TODO find/read http://www.winesandvines.com/template.cfm?section=columns_article&content=62085&columns_id=24

  \textbf{Fermaid K:} Provides micro-nutrients and YAN for yeast health. Pyridoxine and Patothenate are two of 
   these micro-nutrients. 1g/Gallon provides 25ppm of YAN.

% TODO add section on superfood and other nutrients

  \textbf{Yeast Hulls:} also called Yeast Ghosts, are dried yeast cells. Used to aid in unsticking of a stuck 
   fermentation or to combat \ce{H_2S} formation at the end of a fermentation (usually past 2/3 sugar break).
  
% TODO consider adding a subsubsection on pros/cons.

 \subsection{Yeast Feeding}
  The process of yeast nutrient additions is relatively simple at the core. Yeast need nutrients and honey does 
  not have enough, therefore small amounts of nutrients need to be added in order to keep the fermentation healthy. 
  The Nanaiomo Winemakers have additional detail on nutrient addition for fermentation\cite{nanaiomo-fermentation}. 

%TODO figure out a nice way of putting tips and tricks bigger/bolder?
  It is a very good idea to mix powders in liquids that you plan on putting into a fermenting liquid. 
  Adding a powder to a fermenting liquid will introduce nucleation points which will in turn produce lots of foam. 
  Mixing the powder into a liquid allows one to gently stir the mixtur into the must without huge amounts of foam.
  \subsubsection{Guidelines}
   There isn't a set amount of nutrients to add, although there are guidelines. One of the easier (and common) to 
   use nutrient combinations is Fermaid K and DAP at a ratio of 70\% Fermaid K and 30\% 
   DAP\cite{gotmead-fermaid-ratio}. Ratio is based on weight. 
%TODO Add general guidelines for low/med/high musts
%TODO Add Wayne's method?

  \subsubsection{Calculating YAN}
%TODO add link to nanaimo yan calculator
%TODO write up how to do it by hand

 \subsection{Yeast Pitching Rate}
%TODO link to yeast pitching calculators
  Yeast need to be pitched at the proper rate in order to have a high enough cell count to ferment cleanly. 
  There are numerous calculator online, although most
%TODO Is 1.120 right?
  are focused on beer. Generally, one 8g packet of yeast is enough to properly ferment a 5 gallon batch of an 
  under 1.120 O.G. mead. If over 1.120 O.G. pitching two packets is recommended. 

%TODO reference for 1.070 starting gravity
  Sometimes a starter is necessary. The ideal starter for wine yeast is generally 1.070. Using a stir plate is 
  generally a good idea since it oxygenates and stirs, which keeps the yeast in solution and pushes off \ce{CO_2}.
% TODO citation needed

 \subsection{Stirring}
%TODO references on benefits of stirring
%TODO studies if any
  Something as simple as stirring during fermentation can be one of the most important steps in producing an 
  amazing mead. Stirring drives off \ce{CO_2} which is a toxin to yeast. It mixes together the yeast and the must, 
  bring the yeast back into suspension, thus making it more readily able to consume the sugars. It also
  allows a slight mixing in of oxygen which is necessary for yeast growth. Mixing makes it easy to add your 
  nutrients. Having a touch-point with your mead often makes it easy to check for \ce{H_2S} or other issues.

%TODO citation needed over anecdotal evidence
  Stirring tends to make fermentations faster and healthier. 

  There is a higher risk of infection with frequent stirrings. However, if you are careful with sanitation and 
  keep your fermentation area clean, the risk is minimal. The rewards of a proper stirring schedule far outweigh 
  the risks of infection.

  \subsubsection{Stirring Schedule}
%TODO talk to Medsen and others
%TODO references
   The easiest schedule to follow for a traditional mead is every 12 hours until 50\% sugar break or 2/3 sugar 
   break. Stirring much past that tends to introduce oxygen at a stage that can be harmful to the mead. 
   Oxygen at a late stage can begin oxidation of your mead.
   You also want to let the yeast drop at some point so that you can more easily rack.

  \subsubsection{Stirring Schedule with Fruit}

  \subsubsection{Tricks and Tips for Stirring}
%TODO convert to a list?
   Get a drill attached stirring device.

   When beginning to stir, start slowly so as not to be overwhelmed by foam. 

   Put a Star-San soaked rag around the top of the carboy or under the drill to help prevent infection or particles 
   from the drill dropping into your mead.

 \subsection{\ce{H_2S} and Mercaptans}
  One of the simplest indicators of yeast stress is the production of \ce{H_2S}. \ce{H_2S} smells of rotten eggs or 
  sulfur. Yeast stress indicates that there is something off in the fermentation dynamics. The most common reason 
  for \ce{H_2S} is low nutrients. Other causes are temperature (either too low or too high for the yeast) and pH 
  (too low or too high).

  \subsubsection{Eliminating Before 2/3 Sugar Break}
   The closer you get to the 2/3 sugar break mark the more careful you have to be adding inorganic nitrogen (DAP). 
   It may be safer to consider this section as good before 1/2 sugar break, and sometimes helpful between 1/2 sugar
   break and 2/3 sugar break.

   The easiest thing to add here is DAP. If you've added no DAP or Fermaid K you should add a 70/30 mix of Fermaid K
   and DAP. Add an amount to increase YAN by 50ppm, stir and the 
   \ce{H_2S} should lessen in under 5 minutes. If it doesn't, then add another 50ppm and stir again. It is unlikely
   that adding YAN at a total dosage higher than 450ppm will be helpful to the yeast. 
   It is most likely to add off flavors to the final mead as the yeast will either be unable to metabolize it, 
   or metabolize it too quickly resulting in fusels. 
   Notable exceptions where high YAN rate can be helpful are high gravity and high sugar musts.
   If, at this point, it is still smelling of \ce{H_2S} then the issue is likely not because of lack of nutrients, 
   but rather from a low pH, low/high temperature, insufficent \ce{O_2}, or insufficient yeast pitching rate. 
   It should be noted that it is often quite easy to test for pH or temperature before adding nutrients.

  \subsubsection{Eliminating After 2/3 Sugar Break} 
%TODO get references for dosing rates, mixing methodologie
   If you smell \ce{H_2S} after 2/3 sugar break you should not use DAP or fermaid K to remidy the situation. Both of
   those contain inorganic nitrogen (DAP) that the yeast can't metabolize very well at that point of the 
   fermentation. The best thing to add is Yeast Hulls at 0.2g to 0.9g per Gallon. Start with 0.2g/Gallon, and
   then, if after stirring and waiting 5 minutes the \ce{H_2S} smell doesn't go away, add additional Yeast Hulls 
   in 0.2g/Gallon increments (stirring and waiting inbetween additions).

  \subsubsection{Eliminating After Fermentation is Complete}
   If all else fails, and there's still \ce{H_2S} in your mead after the fermentation is done you still have a 
   chance of fixing the mead. It is more difficult and must be caught and fixed quickly. The earlier you catch 
   the \ce{H_2S} and deal with it, the less mercaptans you'll have in your mead. 

   Grapestompers\cite{grapestompers-H2S}
   recommends sulfiting and doing what is often called  ``Splash Racking.'' This is a technique to aerate the 
   sulfited mead to blow off the \ce{H_2S}. If that's not sufficient to remove the \ce{H_2S} it is recommended 
   to rack the mead over copper. 

   The British Columbia Amatuer Winemakers Association\cite{bcawa-H2S} recommends
   aerating and racking the wine through a one inch PVC pipe with copper pot scrubbers inside, mentioning that 
   the surface area is critical. They also do not recommend waiting, but doing the aeration and copper treatment 
   as soon as possible as \ce{H_2S} is converting to sulfides and then disulfides within 2 days of the 
   beginning of \ce{H_2S} production.

% TODO find reference for shelf life decreasing with Copper
   You will also likely want to fine and/or filter your mead after using copper as some report a decrease in 
   shelf life of beverages treated with Copper[CITATION WANTED]. Fining can also reduce the `dull' flavor
   sometimes introduced by adding Copper.

 \subsection{Temperature control}
  This is a very important part of a proper fermentation. In general you want to keep the temperature to the 
  bottom 1/3 of the temperature range of the chosen yeast. This tends to produce the cleanest flavor profile with 
  the least amount of fusels and other off flavors. Increasing the temperature tends to speed up the fermantation.
  It can also help kickstart struggling fermentations. If the temperature is too low fermentation will slow, or 
  even completely stop.

 \subsection{Oxygenation}
% TODO

 \subsection{Must pH}
\textbf{[This section on pH is a work in progress and may contain wrong information]}

%TODO Add some details, references
%Hopefully Wayne can help with some of this.
  Important in maintaining a healthy fermentation. Keeping the pH between 3.7 and 4.6 for initial fermentation is 
  key.

  \subsubsection{\ce{H_2S} or Sluggish Fermentation}
   If you're finding that your mead is producing a large amount of \ce{H_2S} and that temperature, and/or Yeast 
   Hull or nutrient additions are not eliminating it, chances are your pH is off. You will need to measure your 
   pH to see if that is the cause. If you can measure your pH and it is
%TODO Is 3.5 the right number? memory isn't great, look up!
%TODO What is the dosing rate?
%TODO is 3.7 right? find references
%TODO is 4.6 right?
   below 3.5 then you'll want to add some \ce{CaCO_3} (Calcium Carbonate) at [DOSING RATE] until you've reached
   a pH of 3.7. If your pH is too high, above 4.8, 
%TODO list issues.
   you can add some acid blend to adjust below 4.6. There are issues with using too much acid blend or \ce{CaCO_3}.

   If your temperatures are outside the recommended yeast temperature range, slowly adjust the temperatures up or 
   down with a water bath with ice to cool or 
%TODO add links to such devices?
   warm water and/or a heating device such as a fermwrap or brewbelt. You'll want to slowly adjust the temperature, 
   going fast will likely shock the yeast which
%TODO find rate.
   could stop fermentation. Generally, changing the temperature of the must/mead faster than [RATE] can make the 
   yeast drop out and end fermentation.

 \subsection{Step Feeding}
  \subsubsection{Classical Step Feeding}
  \subsubsection{Bottom Dwelling Continuous Diffusion Yeast Feeding (BDC DYF)}