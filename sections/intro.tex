\section{Introduction}
 This is meant to consolidate knowledge of a more advanced nature. Thus this is more for those already somewhat 
 versed in fermentation and mead. There will be some information that seems elementary, and some that is obtuse.

 This is a work in progress and will likely never be completely finished. There may even be information in
 here that isn't entirely correct. As such references will be used whenever found and updates done with more 
 information. Disputed sections will be marked and updated with knowledge as acquired. You are encouraged to 
 contribute. You can do this by submitting issues, asking questions, adding references, or adding content.

 One of the best things you can do to make great mead is to use the best ingredients -- the best honey, the best 
 fruits, and the best spices. Skimping on one or more of your ingredients often results in meads less than the 
 best you can do. Even the best process can't elevate poor ingredients to greatness, however we wish it might.

 To re-iterate, the best thing you can do when making mead is use the highest quality ingredients, and not mess
 up too much. If you read nothing else, read the prior sentence again. Doing a moderate mistake with OK ingredients
 compared to the same mistake with good ingredients won't taste much different in the short term. However in a few
 years, the better ingredients are usually going to shine through. This is compounded with the more ingredients you
 use. If you have 10 ingredients, then every single one has to be top notch, or you're pulling down the others.
 This alone is why doing a traditional mead (honey, water, nutrients, yeast) is an excellent way to start your
 meadmaking journey. There are less thing to vet before usage and you can see where honey alone can really shine.

 You can make good mead with excellent ingredients and OK process. You can make excellent mead with good process. You
 can make amazing mead with excellent process. For the rest of this document the assumption is of using the highest
 quality ingredients and going for better and better processes. This is a living document that will likely never 
 be entirely complete, so newer ideas and processes will be added as found.
 There will be also be some notes on what makes a `high quality ingredient.'

 