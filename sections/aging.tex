\section{Aging}
 This is the process of letting the mead mellow, integrate, and mature. In most cases you want to combine both bulk aging
 and bottle aging to get the best result. You can bulk age in oak or not. Oak imparts additional characteristics such as 
 structure and complexity\cite{oskaar-in-a-nutshell}. 
 A rule of thumb is when the mead doesn't seem to be changing much in bulk, it is time to consider bottling.
 \subsection{Bulk Aging}
  The process of letting finished mead sit in large containers. Primary benefits include\cite{oskaar-in-a-nutshell}: 
  clarification; 
  slower temperature swings due to volume;
  more gradual esterifcation, phenolic evolution, oxydation, oak character extraction (if used or stored in oak);
  ease of management to measure, test, and store.

  Long term aging in carboys requires checking airlocks regularly or risking the entire batch due to oxidation or 
  infection. It easier to get vented stoppers (commonly silicone) for both barrels and carboys instead. These stoppers,
  while more expensive than bungs with stoppers, do not have the maintenance of airlocks.
 \subsection{Bottle Aging}
  Generally aging for 3 months in bottles is good minimum.
% TODO why aging 3 months
% TODO finish flushing out
% Why I like bottles:
% 1. Easier to move than barrels and large vessels
% 2. Long term storage when space is limited
% 3. Continue the benefits of ageing on a bottle by bottle basis
% 4. Easier to taste and evaluate than thieving a sample and sanitizing

