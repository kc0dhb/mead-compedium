\section{Recipes}
 \subsection{General Process}
  Mix your honey with some warm water (in order to make the honey easier to work with). Gradually add cool water 
  until you've reached your volume. Make sure the honey and water is completely mixed. At this point take a 
  gravity reading. If you're shooting for a gravity add honey or water if you're gravity is low or high 
  respectively. Keep in mind that the sugar in honey varies from year to year and hive to hive.

  At this point you should create your 109\textdegree F water for rehydrating your yeast. You will also want to 
  use GO-FERM when hydrating. Mix the water and GO-FERM, then add the yeast and let sit for 15-20 minutes. 
  You will want to cover the yeast with sanitized tinfoil or something to prevent contamination.

%TODO give the YAN calculation stuff.
  Calculate your needed Fermaid-K and DAP additions. For 250ppm YAN (good for use with D47 where it'll go 
  to 14\% potential) you'll need 20.8g Fermaid-K and 9g of DAP.
  You'll add 16.6g Fermaid-K and 4.5g DAP at the end of the lag phase (when the yeast starts bubbling), and 
  then the rest about 24 hours later. 
  There are many schedules that spread the additions out over a longer time. 
  Some claim that these are more effective. 
  You do not want to add DAP (and Fermaid-K contains some DAP) past 66\% sugar break (where 66\% of the sugars 
  have been consumed by the yeast).

  Stirring is critical. It mixes up the yeast and honey mixture, blowing off toxic \ce{CO_2}. 
  It stabalizes the temperature and gives a good chance to check for \ce{H_2S}. And it introduces just a little 
  bit of \ce{O_2} which is needed for yeast growth.
  Stirring should be done twice a day until 66\% sugar break. If you have fruit in your mead you will want to do 
  what is known as punching down the cap.
  This basically means you submerge the fruit layer that is sitting on top. Heat builds up in the fruit cap and can 
  result in nasty off flavors from dying yeast.
  The cap should be punched down about 3 times a day. 
  You want to prevent the cap from getting dry or caking up, as this promotes off flavors such as mold or other 
  bacterial growth. 
  Thus the exact number of times per day depends on the speed of fermentation (faster means more frequent punching 
  of the cap) and the temperature of the fermentation (higher means more frequent).
  This can be accomplished with a quick stir, or with a sanitized spoon (if in a bucket for primary). 
  Larger operations are able to use pumps or horizonal axis stirring methods such as a rotary tank.

  Three useful resources for when you are crafting a recipe are the Gotmead Mead Calculator\cite{gotmead-calculator},
  the Lallemand Yeast Chart\cite{lallemand-yeast},
  and the Nanaiomo Winemakers YAN Yeast Calculator\cite{nanaimo-calculator}.
  
  
 \subsection{Dry Traditional}

  \subsubsection*{Ingredients}
   \begin{tabular}{ d  l }
    12-14 & lbs. Quality Honey \\
    5 & gal. water (to volume)\\
    200-250 & ppm YAN, depends on yeast chosen, 7:3::Fermaid-K:DAP\\
    1 & pkt. re-hydrated Yeast (D-47, DV10, 71B-1122, K1V-1116...)\\
   \end{tabular}

  \subsubsection*{Notes}
   This will make a standard strength dry mead. You'll want to shoot for around 1.085-1.100 for original gravity
   and look for a final gravity between 0.992 an 1.004 with final alcohol around 12-14\% ABV.

 \subsection{Semi-Sweet Traditional}

  \subsubsection*{Ingredients}
   \begin{tabular}{ d  l }
    15-17 & lbs. Quality Honey \\
    5 & gal. water (to volume)\\
    200-250 & ppm YAN, depends on yeast chosen, 7:3::Fermaid-K:DAP\\
    1 & pkt. re-hydrated Yeast D-47\\
   \end{tabular}

  \subsubsection*{Notes}
   This will make a standard strength semi-sweet mead. You'll want to shoot for around 1.110-1.120 for original 
   gravity and look for a final gravity between 1.004 and 1.014 with final alcohol around 14\% ABV. 1.004 is 
   usually considered dry, but there can be some perceptual overlap. If it goes dry, you can always backsweeten 
   by adding more honey. The reason this uses D-47 is because it is a 14\% alcohol tolerant yeast and many of the 
   others will take this all the way dry. If you use a stronger yeast, you may need to add more honey or 
   backsweeten.

 \subsection{Sweet Traditional}

  \subsubsection*{Ingredients}
   \begin{tabular}{ d  l }
    18-21 & lbs. Quality Honey \\
    5 & gal. water (to volume)\\
    200-300 & ppm YAN, depends on yeast chosen, 7:3::Fermaid-K:DAP\\
    1 & pkt. re-hydrated Yeast D-47\\
   \end{tabular}

  \subsubsection*{Notes}
   This will make a standard strength semi-sweet mead. You'll want to shoot for around 1.130-1.150 for original 
   gravity and look for a final gravity between 1.025 and 1.045 with final alcohol around 14\% ABV. The reason 
   this uses D-47 is because it is a 14\% alcohol tolerant yeast and many of the others will take this into 
   semi-sweet or dry territory or into higher alcohol (and longer aging). If you use a stronger yeast, you may 
   need to start with a higher gravity or backsweeten.

 \subsection{Semi-Sweet Session Mead}
TODO

 \subsection{Session Dry-Hopped Mead}
TODO
