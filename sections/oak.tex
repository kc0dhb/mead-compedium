\section{Oak}
  Oak is  a complex and enjoyable addition to mead. It can add complexity as well as structure. However it can easily
  be overdone. Aging with oak is a bit of an art and is about balancing the flavors of oak and mead. It is also
  possible to drastically change the flavor of a mead simply by aging it in a barrel that was previously used
  for spirits or another type of beverage. Rum, Bourbon, and Whisk(e)y are quite common and can easily overwhelm
  your mead if not careful.

 % \subsection{Benefits}
 % sweetness, tannin, complexity, contrast, complimentary flavors, etc

 \subsection{Types of Oak}
  There are quite a few different types of oak, each giving its own characteristic flavor and aroma.
  In general the denser the oak variety the longer it takes to impart its characteristics. 

 \subsection{Barrels}
  Barrels are the traditional way of aging with oak. The size and type of barrel makes a significant impact on
  aging. Bigger barrels take more time, and smaller barrels less. This is approximately linear with respect to
  surface area per volume. Of course a 100 gallon barrel has less surface area per gallon than a 10 Gallon barrel.
  You can take the surface area of the cylinder and divide by the volume to get a comparison number. 
  Treating the barrel as a cylinder is probably good enough for the approximation.
  % TODO insert some numbers here for comparison
  This number
  basically allows you to compare the approximate speed of aging. That is, if 100 days was a good number for the
  amount of oak characteristics you wanted from that 53 gallon barrel, you'll need considerably less time in a 
  15 Gallon barrel. Maybe as little as one sixth of the time. THe higher the ratio of surface area to volume the more 
  often one should taste the barrel.

  For tasting, it is worth noting that having a way to get a taste without opening the barrel is ideal. This can
  be accomplished with a 316 stainless steel nailed drilled into the end about 1/4 to 1/2 of the way from the bottom
  of the barrel. Most folks recommend a 1.5 inch 4D stainless stell nail (McNaster-Carr #97990A102 seems to be common).

 \subsection{Barrel Alternatives}

 \subsection{When to Use Oak}

 \subsection{Aging with Oak}
  Aging with oak is rather straightforward after deciding that your mead needs some oak. Your goal is to get the
  appropriate balance and blending of oak and mead. For some styles, such as traditional, the oak should be supporting
  and not overpowering. If the flavor is particularly strong or dominant than it isn't a traditional mead anymore.
  Some styles, such as high alcohol sweet meads can take a significantly larger amount of oak compared to low alcohol
  dry meads. So the amount of oak aging can change drastically based on the type of mead you are doing. 

  Another thing to take into consideration is that not everyone appreciates oak in the same amount. It may be worth
  doing a taste testing with others to get a feel for where you land on oak preferences. And as taste changes
  over time, it is probably worth doing more than once.

  Be sure to taste every so often. Once the amount of oak you want is acheived you may think it is time to rack
  and bottle and you've captured that flavor. However that is not entirely the case. 
  First, Over time you will notice that the oak characteristics will fade. Second, you may notice that
  significant sediment can drop out in the bottle. This is often more true with barrel aging. Dealing with the first
  is a bit of an art, the second is just patience.

  To combat the oak flavor fading you can simply aim for a slightly over oaked flavor when racking off of oak. It
  isn't a large amount of time that this takes, often it is only 1-10\% more time. Then rack it away from the oak. 
  This is of course not very precise. Another method is to go to your threshold for oaking, then rack at this point.
  Over time the flavor will change and soften.

 % \subsection{Benefits}
 % \subsection{General Aging}
% Ideas
% types of oka
% oak flavor and aroma characteristics
% barrel additions
% oak substitutes
% TODO spellcheck