\section{Oak}
  Oak is  a complex and enjoyable addition to mead. It can add complexity as well as structure. However it can easily
  be overdone. Aging with oak is a bit of an art and is about balancing the flavors of oak and mead. It is also
  possible to drastically change the flavor of a mead simply by aging it in a barrel that was previously used
  for spirits or another type of beverage. Rum, Bourbon, and Whisk(e)y are quite common and can easily overwhelm
  your mead if not careful.

% TODO sweetness, tannin, complexity, contrast, complimentary flavors, etc
% TODO talk on toast level
% TODO talk on roast level

 \subsection{Types of Oak}
  There are quite a few different types of oak, each giving its own characteristic flavor and aroma.
  In general the denser the oak variety the longer it takes to impart its characteristics. 

% TODO talk on french, american, hungarian, croation

 \subsection{Barrels}
  Barrels are the traditional way of aging with oak. The size and type of barrel makes a significant impact on
  aging. Bigger barrels take more time, and smaller barrels less. This is approximately linear with respect to
  surface area per volume. Of course a 100 gallon barrel has less surface area per gallon than a 10 Gallon barrel.
  You can take the surface area of the cylinder and divide by the volume to get a comparison number. 
  Treating the barrel as a cylinder is probably good enough for the approximation.
  % TODO insert some numbers here for comparison
  This number
  basically allows you to compare the approximate speed of aging. That is, if 100 days was a good number for the
  amount of oak characteristics you wanted from that 53 gallon barrel, you'll need considerably less time in a 
  15 Gallon barrel. Maybe as little as one sixth of the time. The higher the ratio of surface area to volume the more 
  often one should taste the barrel.

  For tasting, it is worth noting that having a way to get a taste without opening the barrel is ideal. This can
  be accomplished with a 316 stainless steel nailed drilled into the end about 1/4 to 1/2 of the way from the bottom
  of the barrel. Most folks recommend a 1.5 inch 4D stainless steel nail (McMaster-Carr \#97990A102 seems to be common).

  \subsubsection{Downsides of Barrels}
   It is worth mentioning that there are a few downsides to barrels. They can leak, be an inconsistent size,
   get moldy, burst, get infected, oxidize, can be awkward to move, and annoying to keep when empty.
   All of these can be dealt with to some degree or other, be it prevention or mitigation.

   Leaking can be dealt with by making sure the rings are tight and the barrels are full. A half full barrel has
   a higher likelihood of leaking as the upper half is drying out and shrinking. An empty barrel will most
   certainly dry out after a period of time. Doing a cold water soak, until it stops leaking through, then 
   a hot water soak will help. Never leaving a barrel empty is the easiest way to deal with this. A full
   barrel is a happy barrel.
   % TODO expound on other options here perhaps?

   Since many barrels are hand made products using natural materials, the amount of variance can be large.
   Most barrels are +/- 10\%, but some Hungarian barrels may have a variance as large as +/- 20\%.
   Because of this variance, it is important to have enough mead on hand to fill a 20\% bigger barrel than you expect.
   Also, don't be that surprised if your barrel is smaller than expected.

   Since barrels are a wet and dark environment, mold is a possibility. The issue becomes more likely in humid
   environments as you may get mold outside the barrel, making it more likely to contaminate the inside.
   It is important to keep your barrels full and, if possible, purged of oxygen.
   That means opening the barrel as seldom as possible is advantageous.
   If mold character develops in high levels the batch is unlikely to be salvageable. You should stop using the barrel
   at this point. However, if you must continue using the barrel you can open the barrel and scrape the mold away.
   This is a time consuming process involving cooperage tools and a bit of practice.

% TODO burst

   Another issue with barrels is that they can become infected. That is, organisms other than saccharomyces may be
   living in the barrel in significant enough numbers to impact flavor. Common ones are brettanomyces, lactobacillus
   and pediococcus. These are the common `bugs' in sour beers. If over 10\% ABV the only likely one is brettanomyces.
   Flavors of these organisms range from earthy, to sour, to `butterscotch', to `barn-yard', and more. That isn't to
   say that none of these bugs make good beverages, just that you need to watch out for them.
   Once any of these infect your barrel, that barrel is `sour' forever. These organisms burrow deep into the oak, so
   removing them is nigh impossible. Either live with them, or retire that barrel. If you've used any plastic with that
   barrel you should probably throw it away, or keep for sour barrels only. Stainless you can sterilize by boiling or
   pasteurizing in an oven. A long bleach/water soak for glass works good as well. Just remember that bleach does bad
   things to stainless. If you have access to peracetic acid (a.k.a. peroxyacetic acid or PAA) then this is also an
   excellent sterilizer for brettanomyces. An important thing to mention is that brettanomyces only needs a handful of
   cells to change the flavor of a batch. It is quite happy fermenting anaerobically.

% TODO oxidize
% TODO empty storage

 \subsection{Barrel Alternatives}
  There are many barrel alternatives. The big positive of barrel alternatives is you can use them in glass
  or stainless, and don't need to worry about many of the downsides of barrels. However, the
  alternatives are not perfect. Barrels are made of all edge grain, where some of the barrel alternatives are
  contain a significant portion of edge grain. Edge grain extracts much faster and can have a higher tannin
  flavor. If you are looking for something as close to a barrel as possible then you will want to get an
  alternative with a high a percentage of edge grain as possible.

  WHen using barrel alternatives you will want to sanitize the oak. The most common ways are by boiling 
  or soaking in spirits. Boiling pulls out some tannins, which can be useful to reduce the impact of 
  end grain from cubes. Soaking in spirits for a while imparts a bit of the spirit flavor into the wood. Once
  soaked for a while you can simply throw the oak into your fermenter. Some add the spirits as well, although
  that can be a little over the top depending on volume.

 \subsection{When to Use Oak}

% TODO add structure
% TODO add balance
% TODO add complexity
% TODO infuse a certain spirit characteristic, rum barrel aged for example

 \subsection{Aging with Oak}
  Aging with oak is rather straightforward after deciding that your mead needs some oak. Your goal is to get the
  appropriate balance and blending of oak and mead. For some styles, such as traditional, the oak should be supporting
  and not overpowering. If the flavor is particularly strong or dominant than it isn't a traditional mead anymore.
  Some styles, such as high alcohol sweet meads can take a significantly larger amount of oak compared to low alcohol
  dry meads. So the amount of oak aging can change drastically based on the type of mead you are doing. 

  Another thing to take into consideration is that not everyone appreciates oak in the same amount. It may be worth
  doing a taste testing with others to get a feel for where you land on oak preferences. And as taste changes
  over time, it is probably worth doing more than once.

  Be sure to taste every so often. Once the amount of oak you want is achieved you may think it is time to rack
  and bottle and you've captured that flavor. However that is not entirely the case. 
  First, Over time you will notice that the oak characteristics will fade. Second, you may notice that
  significant sediment can drop out in the bottle. This is often more true with barrel aging. Dealing with the first
  is a bit of an art, the second is just patience.

  To combat the oak flavor fading you can simply aim for a slightly over oaked flavor when racking off of oak. It
  isn't a large amount of time that this takes, often it is only 1-10\% more time. Then rack it away from the oak. 
  This is of course not very precise. Another method is to go to your threshold for oaking, then rack at this point.
  Over time the flavor will change and soften.

 % \subsection{Benefits}
 % \subsection{General Aging}
% Ideas
% types of oka
% oak flavor and aroma characteristics
% barrel additions
% oak substitutes
% TODO spellcheck