\documentclass{article}

\usepackage{dcolumn}
\usepackage[version=3]{mhchem}
\newcolumntype{s}{D{.}{.}{2.5}}
\newcolumntype{d}{D{.}{.}{2.3}}

\addtolength{\voffset}{-1cm}
\addtolength{\hoffset}{-2cm}
\addtolength{\textwidth}{4cm}
\addtolength{\textheight}{2cm}	
%opening
\title{A Compendium of Mead Knowledge}
\author{
  Kyle Byerly\\
 \and
  Various Gotmead Contributors\\
}


\begin{document}

\maketitle
\newpage

\tableofcontents
\newpage

\section{Fermentation}
 \subsection{Yeast}
 \subsection{Yeast Nutrients}
%TODO update nutrient amounts and provide a bibliography
%TODO Have both g/L and g/Gallon measurements
  Nutrients are used to make up for the lack of proper nutrients in the must, to aid in fermentation speed, or to overcome a stuck fermentation. They are typically added in relatively small amounts throughout the fermentation process. Because of the small amount needed, adding too much can be done easily so some amount of care should be taken when calculating and adding nutrients to the must.

%TODO find out how much YAN Go-Ferm provides
  \textbf{Go-Ferm:} Used in rehydrating yeast. 1.25g of Go-Ferm per gram of yeast with 17g grams of water. Mix go-ferm and water together and add yeast at re-hydration temperature (typically 104-109F). Provides 

  \textbf{DAP:} Also known as Di-ammonium phosphate, chmical name \ce{(NH_4)_2 HPO_4} is a source of inorganic nitrogen, which provides YAN (Yeast Available Nitrogen) for nitrogen deficient musts (such as honey). 1g/L provides 210ppm of YAN. 1g/Gallon provides 50ppm of YAN. Do not add past 2/3 sugar break. The yeast cannot consume the nutrient at this stage and it will likely result in Urea type aromas and flavors in the finished mead. Do not add to hydrating yeast as DAP is somewhat toxic to re-hydrating yeast.

  \textbf{Fermaid K:} Provides micro-nutrients and YAN for yeast health. Pyridoxine and Patothenate are two of these micro-nutrients. 1g/Gallon provides 25ppm of YAN.

  Yeast Hulls: Boiled yeast.
  
 \subsection{Yeast Feeding}
 \subsection{Step Feeding}
  \subsubsection{Classical Step Feeding}
  \subsubsection{Bottom Dwelling Continuous Diffusion Yeast Feeding (BDC DYF)}

\section{Honey}

\section{Spices}
 \subsection{While Aging}
 \subsection{Tinctures}
  \subsubsection{Alcohol Tinctures}
  \subsubsection{Water Tinctures}

\section{Oak}
 \subsection{Benefits}
 \subsection{Barrels}
 \subsection{Cubes, Chips, and more}

\section{Fruit}

\section{Fermentation Vessels}
 \subsection{Carboys}
 \subsection{Kegs}
 \subsection{Connicals}
 \subsection{Buckets}

\section{Sanitation}

\section{Aging}
 \subsection{Bulk Aging}
 \subsection{Bottle Aging}


\end{document}