\documentclass{article}

\usepackage{dcolumn}
\usepackage[version=3]{mhchem}
\usepackage{url}
% \usepackage[xindy]{glossaries}
% \makeglossary


\newcolumntype{s}{D{.}{.}{2.5}}
\newcolumntype{d}{D{.}{.}{2.3}}

\addtolength{\voffset}{-1cm}
\addtolength{\hoffset}{-2cm}
\addtolength{\textwidth}{4cm}
\addtolength{\textheight}{2cm}	

% \newglossaryentry{2/3sugar}
% {
% 	name=2/3 Sugar Break,
% 	description=When approximately 2/3 of the sugars have been consumed. Usually an estimate based on O.G. and alcohol tolerance of the yeast.
% }


%opening
\title{A Compendium of Mead Knowledge}
\author{
  Kyle Byerly\\
 \and
  Various Gotmead Contributors\\
}


%TODO decide on tone? 1) scientific. 2) conversational/instructional
\begin{document}

\maketitle
\newpage

\tableofcontents
\newpage

\section{Fermentation}
 This is perhaps the most important aspect of mead making. You can have the best ingredients, but if you don't have a healthy fermentation your honey, and other ingedients, won't shine.
 \subsection{Yeast}
%TODO add temperature ranges for all yeasts.
  There is a vast variety of high quality yeasts available to the average meadmaker. Highlighed below are a few of the favorite more easily available ones.
  In general, the best results from a given yeast are usually obtained when fermenting in the lower third of the yeast's fermentation temperature range.

%TODO add alcohol tolerances.
%TODO surlee spelling?
  D-47: A classic yeast for white wines. Often used to great effect in traditionals. Be careful of letting it get to warm (above 70F) as it throws off fusels and harsh alcohols. It can surlee for quite some time, adding floral characteristics.

%TODO double check spellings
%TODO and details on good for stuff
  DV10: A champagne isolate of a bayanus strain that produces minimal esters and phenols. Also tends to not blow off more delicate honey aromas. 
  Ferments to 18\% regularly. Good for dry meads and high alcohol sweet meads. It is a low nutrient yeast, but tends to produce sulfur unless 
  front-loaded with nutrients (50/25/25 schedule or 75/25 seems to work well).

%TODO specifics?
  71B-1122: Narbonne yeast that can metabolize a certain amount of malic acid. Useful for young meads (low alcohol) or those with a 
  significant portion of an acidic fruit.

  K1V-1116: A generic fruit wine yeast. Seems to produce good results in high temperature fermentation environments such as in excess of 75F.

  Uvaferm 43: A bayanus yeast strain that has an alcohol tolerance of 18\%+. It is commonly used for high alcohol wines and tends not to blow off 
  delicate aromas. Also commonly used with dessert wines and stuck fermentations. 

 \subsection{Yeast Nutrients}
%TODO update nutrient amounts and provide a bibliography
%TODO Have both g/L and g/Gallon measurements
  Nutrients are used to make up for the lack of proper nutrients in the must, to aid in fermentation speed, or to overcome a stuck fermentation. 
  They are typically added in relatively small amounts throughout the fermentation process. Because of the small amount needed, adding too 
  much can be done easily so some amount of care should be taken when calculating and adding nutrients to the must.

%TODO find out how much YAN Go-Ferm provides
  \textbf{Go-Ferm:} Used in rehydrating yeast. 1.25g of Go-Ferm per gram of yeast with 17g grams of water. Mix go-ferm and water together 
  and add yeast at re-hydration temperature (typically 104-109F). Provides 

  \textbf{DAP:} Also known as Di-ammonium phosphate, chmical name \ce{(NH_4)_2 HPO_4} is a source of inorganic nitrogen, which provides YAN 
  (Yeast Available Nitrogen) for nitrogen deficient musts (such as honey). 1g/L provides 210ppm of YAN. 1g/Gallon provides 50ppm of YAN. 
  Do not add past 2/3 sugar break. The yeast cannot consume the nutrient at this stage and it will likely result in Urea type aromas and 
  flavors in the finished mead. Do not add to hydrating yeast as DAP is somewhat toxic to re-hydrating yeast.

  \textbf{Fermaid K:} Provides micro-nutrients and YAN for yeast health. Pyridoxine and Patothenate are two of these micro-nutrients. 
  1g/Gallon provides 25ppm of YAN.

  \textbf{Yeast Hulls:} also called Yeast Ghosts are dried yeast cells. Used to aid in unsticking of a stuck fermentation or to combat 
  \ce{H_2S} formation at the end of a fermentation (usually past 2/3 sugar break).
  
 \subsection{Yeast Feeding}
  The process of yeast nutrient additions is relatively simple at the core. Yeast need nutrients and honey does not have enough, therefore small amounts
  of nutrients need to be added in order to keep the fermentation healthy. The Nanaiomo Winemakers have a good bit of detail on nutrient addition for fermentation
  \cite{nanaiomo-fermentation}. 
  \subsubsection{Guidelines}
   There isn't a set amount of nutrients to add, although there are guidelines. One of the easier (and common) to use nutrient 
   combinations is Fermaid K and DAP at a ratio of 70\% Fermaid K and 30\% DAP. Ratio is based on weight. 
 \subsection{\ce{H_2S} and Mercaptans}
  One of the simplest indicators of yeast stress is the production of \ce{H_2S}. \ce{H_2S} smells of rotten eggs or sulfur. Yeast stress indicates that there is
  something off in the fermentation dynamics. The most common reason for \ce{H_2S} is low nutrients. Other causes are temperature (either too low or too high for 
  the yeast) and pH (too low or too high).

  \subsubsection{Eliminating Before 2/3 Sugar Break} 

  \subsubsection{Eliminating After 2/3 Sugar Break} 
   If you smell \ce{H_2S} after 2/3

 \subsection{Must pH}
 \subsection{Temperate control}

 \subsection{Step Feeding}
  \subsubsection{Classical Step Feeding}
  \subsubsection{Bottom Dwelling Continuous Diffusion Yeast Feeding (BDC DYF)}

\section{Honey}

\section{Spices}
 \subsection{While Aging}
 \subsection{Tinctures}
  \subsubsection{Alcohol Tinctures}
  \subsubsection{Water Tinctures}

\section{Oak}
 \subsection{Benefits}
 \subsection{Barrels}
 \subsection{Cubes, Chips, and more}

\section{Fruit}

\section{Fermentation Vessels}
 \subsection{Carboys}
 \subsection{Kegs}
 \subsection{Connicals}
 \subsection{Buckets}

\section{Sanitation}

\section{Aging}
 \subsection{Bulk Aging}
 \subsection{Bottle Aging}

% \section{Glossary}
% \printglossary

\newpage
%TODO decide if inline or a separate file is better for organization.
\begin{thebibliography}{99}
 \bibitem{nanaiomo-fermentation}
  Nanaiomo Winemakers, Adding Nitrogen to Fermentations.
  \url{http://www.nanaimowinemakers.org/Winemaking/General/AddingNitrogen.htm}

\end{thebibliography}


\end{document}