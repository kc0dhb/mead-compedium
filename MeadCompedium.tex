\documentclass{article}

\usepackage{dcolumn}
\usepackage[version=3]{mhchem}
\usepackage{url}
% \usepackage[xindy]{glossaries}
% \makeglossary


\newcolumntype{s}{D{.}{.}{2.5}}
\newcolumntype{d}{D{.}{.}{2.3}}

\addtolength{\voffset}{-1cm}
\addtolength{\hoffset}{-2cm}
\addtolength{\textwidth}{4cm}
\addtolength{\textheight}{2cm}	

% \newglossaryentry{2/3sugar}
% {
% 	name=2/3 Sugar Break,
% 	description=When approximately 2/3 of the sugars have been consumed. Usually an estimate based on O.G. and alcohol tolerance of the yeast.
% }


%opening
\title{A Compendium of Mead Knowledge}
\author{
  Kyle Byerly\\
 \and
  Various Gotmead Contributors\\
}


%TODO decide on tone? 1) scientific. 2) conversational/instructional
%TODO consume/link/validate parts of the 'Oskaar in a Nutshell' thread
\begin{document}

\maketitle
\newpage

\tableofcontents
\newpage

\section{Fermentation}
 This is perhaps the most important aspect of mead making. You can have the best ingredients, but if you don't have a healthy fermentation your honey, and other ingedients, won't shine.
 \subsection{Yeast}
%TODO add temperature ranges for all yeasts.
  There is a vast variety of high quality yeasts available to the average meadmaker. Highlighed below are a few of the favorite more easily available ones.
  In general, the best results from a given yeast are usually obtained when fermenting in the lower third of the yeast's fermentation temperature range.

%TODO add alcohol tolerances.
%TODO surlee spelling?
  D-47: A classic yeast for white wines. Often used to great effect in traditionals. Be careful of letting it get to warm (above 70F) as it throws off fusels and harsh alcohols. It can surlee for quite some time, adding floral characteristics.

%TODO double check spellings
%TODO and details on good for stuff
  DV10: A champagne isolate of a bayanus strain that produces minimal esters and phenols. Also tends to not blow off more delicate honey aromas. 
  Ferments to 18\% regularly. Good for dry meads and high alcohol sweet meads. It is a low nutrient yeast, but tends to produce sulfur unless 
  front-loaded with nutrients (50/25/25 schedule or 75/25 seems to work well).

%TODO specifics?
  71B-1122: Narbonne yeast that can metabolize a certain amount of malic acid. Useful for young meads (low alcohol) or those with a 
  significant portion of an acidic fruit.

  K1V-1116: A generic fruit wine yeast. Seems to produce good results in high temperature fermentation environments such as in excess of 75F.

  Uvaferm 43: A bayanus yeast strain that has an alcohol tolerance of 18\%+. It is commonly used for high alcohol wines and tends not to blow off 
  delicate aromas. Also commonly used with dessert wines and stuck fermentations. 

 \subsection{Yeast Nutrients}
%TODO update nutrient amounts and provide a bibliography
%TODO Have both g/L and g/Gallon measurements
  Nutrients are used to make up for the lack of proper nutrients in the must, to aid in fermentation speed, or to overcome a stuck fermentation. 
  They are typically added in relatively small amounts throughout the fermentation process. Because of the small amount needed, adding too 
  much can be done easily so some amount of care should be taken when calculating and adding nutrients to the must.

%TODO find out how much YAN Go-Ferm provides
  \textbf{Go-Ferm:} Used in rehydrating yeast. 1.25g of Go-Ferm per gram of yeast with 17g grams of water. Mix go-ferm and water together 
  and add yeast at re-hydration temperature (typically 104-109F). Provides 

  \textbf{DAP:} Also known as Di-ammonium phosphate, chmical name \ce{(NH_4)_2 HPO_4} is a source of inorganic nitrogen, which provides YAN 
  (Yeast Available Nitrogen) for nitrogen deficient musts (such as honey). 1g/L provides 210ppm of YAN. 1g/Gallon provides 50ppm of YAN. 
  Do not add past 2/3 sugar break. The yeast cannot consume the nutrient at this stage and it will likely result in Urea type aromas and 
  flavors in the finished mead. Do not add to hydrating yeast as DAP is somewhat toxic to re-hydrating yeast.

  \textbf{Fermaid K:} Provides micro-nutrients and YAN for yeast health. Pyridoxine and Patothenate are two of these micro-nutrients. 
  1g/Gallon provides 25ppm of YAN.

  \textbf{Yeast Hulls:} also called Yeast Ghosts are dried yeast cells. Used to aid in unsticking of a stuck fermentation or to combat 
  \ce{H_2S} formation at the end of a fermentation (usually past 2/3 sugar break).
  
 \subsection{Yeast Feeding}
  The process of yeast nutrient additions is relatively simple at the core. Yeast need nutrients and honey does not have enough, therefore small amounts
  of nutrients need to be added in order to keep the fermentation healthy. The Nanaiomo Winemakers have a good bit of detail on nutrient addition for fermentation
  \cite{nanaiomo-fermentation}. 

%TODO figure out a nice way of putting tips and tricks bigger/bolder?
  It is a very good idea to mix powders in liquids that you plan on putting into a fermenting liquid. Adding a powder to a fermenting liquid will introduce nucleation
  points which will in turn produce lots of foam. Mixing the powder into the foam allows one to gently stir the mix into the must without huge amounts of foam.
  \subsubsection{Guidelines}
   There isn't a set amount of nutrients to add, although there are guidelines. One of the easier (and common) to use nutrient 
   combinations is Fermaid K and DAP at a ratio of 70\% Fermaid K and 30\% DAP. Ratio is based on weight. 
%TODO Add general guidelines for low/med/high musts
%TODO Add Wayne's method?

  \subsubsection{Calculating YAN}
%TODO add link to nanaimo yan calculator
%TODO write up how to do it by hand

 \subsection{Yeast Pitching Rate}
%TODO link to yeast pitching calculators
  Yeast need to be pitched at the proper rate in order to have a high enough cell count to ferment cleanly. There are numerous calculator online, although most
%TODO Is 1.120 right?
  are focused on beer. Generally one 8g packet of yeast is enough to properly ferment a 5 gallon batch of an under 1.120 O.G. mead. If over than 1.120 O.G. pitching
  two packets is recommended. 

%TODO reference for 1.070 starting gravity
  Sometimes a starter is necessary. The ideal starter for wine yeast is generally 1.070. Using a stir plate is generally a good idea since it oxygenates and stirs,
  which keeps the yeast in solution and pushes off \ce{CO_2}.

 \subsection{Stirring}
%TODO references on benefits of stirring
%TODO studies if any
  Something as simple as stirring during fermentation can be one of the most important steps in producing an amazing mead. Stirring drives off \ce{CO_2} which is a 
  toxin to yeast. It mixes together the yeast and the must, bring the yeast back into suspension, thus making it more readily able to consume the sugars. It also
  allows a slight mixing in of oxygen which is necessary for yeast growth. Mixing makes it easy to add your nutrients. Having a touch-point with your mead often
  make is easy to check for \ce{H_2S} or other issues.

%TODO citation needed over anecdotal evidence
  Stirring tends to make fermentations faster and healthier. 

  There is a higher risk of infection with frequent stirrings. However, if you are careful with sanitation and keep your fermentation area clean the risk is minimal.
  The rewards of a proper stirring schedule far outweigh the risks of infection.

  \subsubsection{Stirring Schedule}
%TODO talk to Medsen and others
%TODO references
   The easiest schedule to follow for a traditional mead is every 12 hours until 50\% sugar break or 2/3 sugar break. Stirring much past that tends to introduce
   oxygen at a stage that oxygen is less than ideal. You also want to let the yeast drop at some point.

  \subsubsection{Stirring Schedule with Fruit}

  \subsubsection{Tricks and Tips for Stirring}
%TODO convert to a list?
   Get a drill attached stirring device.

   When beginning to stir, start slowly so as not to be overwhelmed by foam. 

   Put a Star-San soaked rag around the top of the carboy or under the drill to help prevent infection or particles from the drill dropping into your mead.

  \subsection{Oxygenation}

 \subsection{\ce{H_2S} and Mercaptans}
  One of the simplest indicators of yeast stress is the production of \ce{H_2S}. \ce{H_2S} smells of rotten eggs or sulfur. Yeast stress indicates that there is
  something off in the fermentation dynamics. The most common reason for \ce{H_2S} is low nutrients. Other causes are temperature (either too low or too high for 
  the yeast) and pH (too low or too high).

  \subsubsection{Eliminating Before 2/3 Sugar Break}
%TODO add proper Rate 
   The easiest thing to add here is DAP. If you've added no DAP or Fermaid K you should add a 70/30 mix of Fermaid K and DAP. Add at a dosing rate of [RATE], stir and the 
%TODO add proper secondary rate.
   \ce{H_2S} should lessen in under 5 minutes. If it doesn't, then add [RATE] and stir again. If at this point it is still smelling of \ce{H_2S} then the issue is likely
   not because of lack of nutrients, but rather from a low pH, low temperature, or insufficient yeast pitching rate.

  \subsubsection{Eliminating After 2/3 Sugar Break} 
%TODO get references for dosing rates, mixing methodologie
   If you smell \ce{H_2S} after 2/3 sugar break you should not use DAP or fermaid K to remidy the situation. Both of those contain inorganic nitrogen (DAP) 
   that the yeast can't metabolize very well at that point of the fermentation. The best thing to add is Yeast Hulls at .5g to .9g per Gallon. Start with .5g/Gallon and
   then if after stirring and waiting 5 minutes the \ce{H_2S} smell doesn't go away add additional Yeast Hulls in .2g/Gallon increments (stirring and waiting inbetween
   additions).

 \subsection{Temperate control}

 \subsection{Must pH}
%TODO Add some details, references
%Hopefully Wayne can help with some of this.
  Important in maintaining a healthy fermentation. Keeping the pH between 3.7 and 4.6 for initial fermentation is key.

  \subsubsection{\ce{H_2S} or Sluggish Fermentation}
   If you're finding that your mead is producing a large amount of \ce{H_2S} and that temperature, and/or Yeast Hull or nutrient additions are not eliminating it, 
   chances are your pH is off. You will need to measure your pH to see if that is the cause. If you can measure your pH and it is
%TODO Is 3.5 the right number? memory isn't great, look up!
%TODO What is the dosing rate?
%TODO is 3.7 right? find references
%TODO is 4.6 right?
   below 3.5 then you'll want to add some \ce{CaCO_3} (Calcium Carbonate) at [DOSING RATE] until you've reached a pH of 3.7. If your pH is too high, above 4.8, 
%TODO list issues.
   you can add some acid blend to adjust below 4.6. There are issues with using too much acid blend or \ce{CaCO_3}.

   If your temperatures are outside the recommended yeast temperature range, slowly adjust the temperatures up or down with a water bath with ice to cool or 
%TODO add links to such devices?
   warm water and/or a heating device such as a fermwrap or brewbelt. You'll want to slowly adjust the temperature, going fast will likely shock the yeast which
%TODO find rate.
   could stop fermentation. Generally changing the temperature of the must/mead faster than {RATE] can make the yeast drop out and end fermentation.

 \subsection{Step Feeding}
  \subsubsection{Classical Step Feeding}
  \subsubsection{Bottom Dwelling Continuous Diffusion Yeast Feeding (BDC DYF)}

\section{Honey}

\section{Spices}
 \subsection{While Aging}
 \subsection{Tinctures}
  \subsubsection{Alcohol Tinctures}
  \subsubsection{Water Tinctures}

\section{Oak}
 \subsection{Benefits}
 \subsection{Barrels}
 \subsection{Cubes, Chips, and more}

\section{Fruit}

\section{Fermentation Vessels}
 \subsection{Carboys}
 \subsection{Kegs}
 \subsection{Connicals}
 \subsection{Buckets}

\section{Sanitation}

\section{Aging}
 \subsection{Bulk Aging}
 \subsection{Bottle Aging}

% \section{Glossary}
% \printglossary

\newpage
%TODO decide if inline or a separate file is better for organization.
\begin{thebibliography}{99}
 \bibitem{nanaiomo-fermentation}
  Nanaiomo Winemakers, Adding Nitrogen to Fermentations.
  \url{http://www.nanaimowinemakers.org/Winemaking/General/AddingNitrogen.htm}

\end{thebibliography}


\end{document}